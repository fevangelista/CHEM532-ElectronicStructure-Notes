% This work is licensed under the Creative Commons Attribution-NonCommercial 4.0 International License.
% To view a copy of this license, visit http://creativecommons.org/licenses/by-nc/4.0/
% or send a letter to Creative Commons, PO Box 1866, Mountain View, CA 94042, USA.

% !TEX TS-program = xelatex	

\documentclass[../Main/chem532-notes.tex]{subfiles}

\begin{document}

\setcounter{chapter}{9}

\chapter{Excited state methods}

In this chapter we will go over the main approaches used to compute electronically excited states.
The problem of describing excited states is similar to that of computing ground states with pronounced static correlation effects.
However, in many cases one can assume that the ground state is well described by a single Slater determinants or by ground state DFT, and this allows us to employ the formalism of single-reference methods as a way to build excited state methods.
In other cases, both the ground and its excited states may display static correlation, in which case it is necessary to use a multireference method to describe both the ground and the excited states.
This is particularly important when one is interested in computing states near a conical intersection, where two states become degenerate.

\section{Configuration interaction singles}
The simplest excited state method is configuration interaction with singles (CIS).
The CIS wave function is a linear combination of the Hartree--Fock determinant plus all of its singly excited determinants
\begin{equation}
\ket{\Psi_\mathrm{CIS}} = c_0 \ket{\Phi_0}  + \sum_{i}^{\mathrm{occ}}\sum_{a}^{\mathrm{vir}} c_i^a \cop{a}\aop{i} \ket{\Phi_0}.
\end{equation}

\section{Time-dependent DFT}

\section{Equation-of-motion coupled cluster methods}

\section{Equation-of-motion coupled cluster theory}
\begin{equation}
\hat{H} \rightarrow \bar{H} = e^{-\hat{T}} \hat{H} e^{\hat{T}} 
\end{equation}
If 
\begin{equation}
\hat{H}  \ket{\Psi_\alpha} = E_\alpha \ket{\Psi_\alpha}
\end{equation}

\begin{equation}
\begin{split}
\hat{H}  e^{\hat{T}} e^{-\hat{T}} \ket{\Psi_\alpha} = & E_\alpha \ket{\Psi_\alpha} \\
e^{-\hat{T}} \hat{H}  e^{\hat{T}} e^{-\hat{T}} \ket{\Psi_\alpha} = & E_\alpha e^{-\hat{T}} \ket{\Psi_\alpha} \\
\bar{H} e^{-\hat{T}} \ket{\Psi_\alpha} = & E_\alpha e^{-\hat{T}} \ket{\Psi_\alpha} \\
\bar{H} \ket{\bar{\Psi}_\alpha} = & E_\alpha \ket{\bar{\Psi}_\alpha} \\
\end{split}
\end{equation}


\section{Multireference methods for excited states}




\section{CASSCF methods for excited states}
State-specific CASSCF

\begin{equation}
\ket{\Psi_\text{SS-CASSCF}^{(\alpha)}} 
= \sum_I C_I^{(\alpha)} \ket{\Phi^{(\alpha)}_I}
\end{equation}
where
\begin{equation}
\ket{\Phi^{(\alpha)}_I} =  \ket{\psi^{(\alpha)}_{i_1} \cdots \psi^{(\alpha)}_{i_N}}
\end{equation}


\begin{equation}
\ket{\Psi_\text{SA-CASSCF}^{(\alpha)}} 
= \sum_I C_I^{(\alpha)} \ket{\Phi_I}
\end{equation}

\begin{equation}
\bar{E} = \min_{C_I^{\alpha}, \{ \psi_i \}} \sum_{\alpha = 1}^{d}
w_\alpha \braket{\Psi_\text{SA-CASSCF}^{(\alpha)} | \hat{H} | \Psi_\text{SA-CASSCF}^{(\alpha)}} 
\end{equation}

\end{document}
